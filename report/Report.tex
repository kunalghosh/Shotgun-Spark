% This is a model template for the solutions in computational science. You can find a very useful documentation for LaTeX in Finnish at ftp://ftp.funet.fi/pub/TeX/CTAN/info/lshort/finnish/ or in English at ftp://ftp.funet.fi/pub/TeX/CTAN/info/lshort/english/. The section List of mathematical symbols in Chapter 3 is especially useful for the typesetting of mathematical formulas.

% Compile the document to PDF by command 'pdflatex model.tex' in the terminal. The command must be run twice for the references in the text to be correct.

\documentclass[a4paper,11pt]{article}
\usepackage[utf8]{inputenc}
% This includes letters such as � and �
\usepackage[T1]{fontenc}
% Use here 'Finnish' for Finnish hyphenation. You may have to compile the code twice after the change. 
\usepackage[english]{babel}
\usepackage{graphicx}
% Some math stuff
\usepackage{amsmath,amsfonts,amssymb,amsbsy,commath,booktabs}  
% This is just to include the urls
\usepackage{hyperref}
\usepackage[margin=2cm]{geometry}

\setlength{\parindent}{0mm}
\setlength{\parskip}{1.0\baselineskip}

\usepackage{listings}
\usepackage{color}

\definecolor{dkgreen}{rgb}{0,0.6,0}
\definecolor{gray}{rgb}{0.5,0.5,0.5}
\definecolor{mauve}{rgb}{0.58,0,0.82}

\lstset{frame=tb,
	language=Python,
	aboveskip=3mm,
	belowskip=3mm,
	showstringspaces=false,
	columns=flexible,
	basicstyle={\small\ttfamily},
	numbers=none,
	numberstyle=\tiny\color{gray},
	keywordstyle=\color{blue},
	commentstyle=\color{dkgreen},
	stringstyle=\color{mauve},
	breaklines=true,
	breakatwhitespace=true,
	tabsize=4
}

\begin{document}

\title{Convex Optimization for Big Data \\ Implementing Shotgun in the "Cloud"} % Replace the exercise round number
\author{Kunal Ghosh, 546247 // Jussi Ojala, 544605} % Replace with your name and student number
\maketitle

\section{Objective}\label{prob1}
The objective of this assignment is to implement the "Shotgun"
algorithm for parallel coordinate descent. We already had a Matlab
implementation as a part of the course work and In this part of the course we aim to do the following.
\begin{itemize}
    \item Implement the "Shotgun" algorithm in Apache Spark using the pySpark API.
    \item Document the setup of Amazon Elastic Map Reduce (EMR) cloud infrastructure with all the necessary libraries to run pySpark code.
    \item Document the parts of the algorithm which were not available in pyspak.mllib.linalg.distributed module and we had to implement ourselves. This is to ensure these code snippets can be used in other projects.
    \item Run the pySpark Implementation of the "Shotgun" algorithm on a few large datasets and do preliminary analysis of the results.
\end{itemize}
To summarize, we want this assignment to be a guide on how to \textbf{speed-up optimization algorithm on the cloud} rather how to implement one particular optimization algorithm "Shotgun" using Apache spark.
\end{document}
